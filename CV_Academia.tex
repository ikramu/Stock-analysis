%%%%%%%%%%%%%%%%%%%%%%%%%%%%%%%%%%%%%%%%%
% Medium Length Graduate Curriculum Vitae
% LaTeX Template
% Version 1.1 (9/12/12)
%
% This template has been downloaded from:
% http://www.LaTeXTemplates.com
%
% Original author:
% Rensselaer Polytechnic Institute (http://www.rpi.edu/dept/arc/training/latex/resumes/)
%
% Important note:
% This template requires the res.cls file to be in the same directory as the
% .tex file. The res.cls file provides the resume style used for structuring the
% document.
%
%%%%%%%%%%%%%%%%%%%%%%%%%%%%%%%%%%%%%%%%%

%----------------------------------------------------------------------------------------
%	PACKAGES AND OTHER DOCUMENT CONFIGURATIONS
%----------------------------------------------------------------------------------------

\documentclass[margin, 10pt]{res} % Use the res.cls style, the font size can be changed to 11pt or 12pt here

\usepackage{helvet} % Default font is the helvetica postscript font
%\usepackage{newcent} % To change the default font to the new century schoolbook postscript font uncomment this line and comment the one above

\usepackage{hyperref}
\hypersetup{
    colorlinks,
    linkcolor={green!50!black},
    citecolor={green!50!black},
    urlcolor={blue!80!black}
}
\usepackage[usenames,dvipsnames]{xcolor}
\usepackage[T1]{fontenc}

\setlength{\textwidth}{5.1in} % Text width of the document

\begin{document}

%----------------------------------------------------------------------------------------
%	NAME AND ADDRESS SECTION
%----------------------------------------------------------------------------------------

\moveleft.5\hoffset\centerline{\huge\textbf{\textsc{Ikram Ullah}} } % Your name at the top
 
\moveleft\hoffset\vbox{\hrule width\resumewidth height 1pt}\smallskip % Horizontal line after name; adjust line thickness by changing the '1pt'
 
\moveleft.5\hoffset\centerline{Studentbacken 21/1515 \hfill \href{mailto:ikramu@kth.se}{ikramu@kth.se}} % Your address
\moveleft.5\hoffset\centerline{G\"ardet, 11557 \hfill \href{tel:+46739366175}{+46 73 936 6175}}%(073) 936-6175}
\moveleft.5\hoffset\centerline{Stockholm \hfill \href{http://www.nada.kth.se/~ikramu/}{http://www.nada.kth.se/$\sim$ikramu/}}
%\moveleft.5\hoffset\centerline{}
%\moveleft.5\hoffset\centerline{}
\moveleft\hoffset\vbox{\hrule width\resumewidth height 1pt}\smallskip % Horizontal line after name; adjust line thickness by changing the '1pt'

%----------------------------------------------------------------------------------------

\begin{resume}

%----------------------------------------------------------------------------------------
%	OBJECTIVE SECTION
%----------------------------------------------------------------------------------------
 
\section{OBJECTIVE}  

To model and implement market-based eco-friendly solutions for challenging problems in healthcare, utilizing sound mathematical principles, latest software and hardware technologies.

%To model and implement mathematically inspired eco-friendly solutions for real world phenomena, with a focus on market-based solutions, using latest hardware and software.

%----------------------------------------------------------------------------------------
%	EDUCATION SECTION
%----------------------------------------------------------------------------------------

\section{EDUCATION}

{\bf \color{Black}{Doctor of Philosophy},} Computer Science \\
{\color{RubineRed}{Kungliga Tekniska H\"{o}gskolan, Sweden}} \hfill \textit{Jun 2010 -- Dec 2014} \\
Concentration: Bioinformatics \\
Minor: Comparative Genomics 

%-----------------------------------------%
{\bf \color{Black}{Exchange Student},} Computer Science \\
{\color{RubineRed}{University of Limerick, Ireland}} \hfill \textit{Mar 2009 -- Aug 2009} \\
Concentration: Software Engineering 

%-----------------------------------------%
{\bf \color{Black}{Master of Science,}} Computer Science (CGPA 3.47 out of 4.0) \\
{\color{RubineRed}{Lahore University of Management Sciences (LUMS), Pakistan}} \hfill \textit{Sep 2006 -- Aug 2009} \\
Concentration: Artificial Intelligence \\
Minor: Software Engineering

%-----------------------------------------%
{\bf \color{Black}{Bachelor of Science,}} Computer Science (CGPA 3.8 out of 4.0) \\
{\color{RubineRed}{University of Peshawar, Pakistan}} \hfill \textit{Jan 2002 -- Mar 2006} \\
Concentration: Computer Science \\
Minor: Mathematics

%----------------------------------------------------------------------------------------
%	RESEARCH SOFTWARE SECTION
%----------------------------------------------------------------------------------------

\section{RESEARCH \\ SOFTWARE} 
\begin{itemize}
\item \textbf{\href{http://prime.scilifelab.se/}{PrIME}} -- PrIME is a C++ library with tools for phylogenetic inference. Computationally, the emphasis is on probabilistic models that typically employ a Markov Chain Monte Carlo (MCMC) based sampling.
\item \textbf{\href{https://code.google.com/p/jprime/}{JPrIME}} -- JPrIME is the Java based implementation of PrIME with computational optimization of existing tools, and inclusion of additional tools for phylogenetic inference.
\item \textbf{\href{http://prime.scilifelab.se/dlrsorthology/}{DLRSOrthology}} -- DLRSOrthology is a program for Bayesian probabilistic orthology analysis using an integrated gene evolution and sequence evolution model as outlined in the \href{http://www.pnas.org/content/106/14/5714.long}{DLRS} model. The implementation is in GNU C++ using Boost and PrIME libraries while the auxiliary scripts are written in R, Python and Perl.
\item \textbf{\href{http://prime.scilifelab.se/mixtreem/}{MixTreEM}} -- MixTreEM is an MPI based parallel implementation of a novel species tree inference algorithm using mixture model. The software is writtten in C++ and Boost-MPI routines are used for making the code parallel.
\end{itemize}

%----------------------------------------------------------------------------------------
%	COMPUTER SKILLS SECTION
%----------------------------------------------------------------------------------------

\section{COMPUTER \\ SKILLS} 

%{\sl \textbf{\color{Brown}{\textbf{Chief Data Scientist}}}}
{\bf \color{Brown}{Languages:}} 
C++, Java, C\#, R, Matlab \\
{\bf \color{Brown}{Misc Tools:}} Bash, CMake, Python, Boost C++ libraries, OpenMPI \& Boost-MPI, MySQL, SQL Server, RapidMiner, Weka, Git, SVN, Netbeans, Eclipse, and different Bioinformatics tools. \\
{\bf \color{Brown}{HPC Clusters:}} Developed software using 4 Swedish Unix clusters (\href{https://www.pdc.kth.se/resources/computers/ferlin}{Ferlin}, \href{https://www.nsc.liu.se/systems/triolith/}{Triolith}, \href{http://www.uppmax.uu.se/}{Tintin}, and \href{http://www.hpc2n.umu.se/resources/abisko}{Abisko}) \\
{\bf \color{Brown}{Operating Systems:}} Unix, Windows and OS X.

%----------------------------------------------------------------------------------------
%	PROFESSIONAL EXPERIENCE SECTION
%----------------------------------------------------------------------------------------
 
\section{PROFESSIONAL \\ EXPERIENCE}

{\sl \textbf{\color{Brown}{\textbf{Chief Data Scientist}}}} \hfill \textit{April 2014 onward} \\
{\color{RubineRed}{Greenely, Sweden}} \\
\href{https://www.youtube.com/watch?v=COsyhfX7Pes}{Greenely} is a startup about eco-friendly intelligent energy management. The idea is to optimize household electricity consumption using machine learning based energy disaggregation algorithms and user behavior mining.
\begin{itemize} 
\item Implementing energy disaggregation algorithms
\item Leading technical team
\item Managing and optimizing the server-side of the software
\item Tools used are R, PostgreSQL, Java and Go.
\end{itemize} 

{\sl \color{Brown}{\textbf{Analyst Software Engineer}}} \hfill \textit{Oct 2009 -- May 2010} \\
{\href{http://www.satmapinc.com/}{\color{RubineRed}{SATMAP Inc, Machine Learning Team, Pakistan}}}  \\
SATMAP is an enterprise call center application used for intelligent call routing used by Fortune 500 companies like AT\&T, Gieco, Time Warners Cables etc. I worked with a team of algorithm designers in:

\begin{itemize} \itemsep -2pt % Reduce space between items
\item Evaluating novel machine learning/data mining algorithms on call center data. 
\item Implementing, testing, and integrating selected algorithms with SATMAP. 
\item Tools used were combination of C++, SQL Server, R, and Rapidminer.
\end{itemize}
 
{\sl \textbf{\color{Brown}{\textbf{Research Assistant}}}} \hfill \textit{Dec 2007 -- Oct 2008} \\
{\color{RubineRed}{Computer Science Department, LUMS, Pakistan}} 
\begin{itemize} 
\item Woking with XVCL: an open source library providing component based Document/View Architecture support for developers. Implementation was mainly in Java.
\item Teaching assistant for Advanced Software Engineering course
\end{itemize} 

{\sl \textbf{\color{Brown}{\textbf{Consulting/Freelancing}}}} \hfill \textit{Mid-2008 onwards} \\
{\color{RubineRed}{Odesk Corporation} \hfill \href{https://www.odesk.com/users/~013a228837c241737c}{Odesk profile link}} \\
I started consulting at rentacoder.com (now acquired by freelancer.com) and then switched to odesk.com, offering consultancy mainly for machine learning projects. Some of these include: 

\begin{itemize} \itemsep -2pt % Reduce space between items
\item Stock value analysis and prediction using association rule mining (using R and Java)
\item Quality based classification of barcode images using computer vision algorithms (using Matlab)
\item Text classification using different machine learning algorithms (using Rapidminer and Java) 
\end{itemize} 
Details on any of above (and those not listed here) can be furnished on demand.

{\sl \textbf{\color{Brown}{\textbf{Software Internee}}}} \hfill \textit{Summers 2007} \\
{\color{RubineRed}{Five Rivers Technologies, Lahore}}
\begin{itemize}
\item Part of the group implementing a dual streaming/synchronization server for mobile devices using Funambol DS Server and video streaming APIs.
\end{itemize} 


%----------------------------------------------------------------------------------------
%	Computer Vision SECTION
%---------------------------------------------------------------------------------------- 
%\section{COMPUTER VISION \\ EXPERIENCE}
%{\sl \textbf{\color{Brown}{\textbf{Eye-based cursor movement}}}} \\
%Using computer vision algorithms to track iris movement in the eye and map it to drive cursor movement on the screen. %The software was implemented in C++ using Intel OpenCV libraries.  \\
%{\sl \textbf{\color{Brown}{\textbf{Classification of barcode images}}}} \\
%Using different image denoising algorithms implemented in Image Processing toolbox in Matlab to classify images to %translatable and non-translatable classes. The solution was used as pre-step in a barcode reader application.  

%----------------------------------------------------------------------------------------
%	Team leading SECTION
%---------------------------------------------------------------------------------------- 
\section{TEAM-LEAD \\ EXPERIENCE}
\begin{itemize}
\item I have managed a small team of developers in Pakistan during my \href{https://www.odesk.com/users/~013a228837c241737c}{consultancy career}. Major responsibilities included task allocation, time allocation, and ensuring team collaboration.
\item Recently, I am leading a group of KTH masters students in \href{https://www.youtube.com/watch?v=COsyhfX7Pes}{Greenely} project. Major responsibilities include task allocation, task synchronization, and team motivation.
\end{itemize}
%----------------------------------------------------------------------------------------
%	Modeling SECTION
%---------------------------------------------------------------------------------------- 

%\section{COURSES}

%Following are some of the courses I took at KTH during my PhD. 
%\begin{itemize}
%\item High performance computing
%\item Advanced machine learning
%\item Graph theory and algorithms
%\item Elements of statistical learning
%\item Scientific writing
%\item Algorithmic bioinformatics
%\item Quantitative system biology
%\end{itemize} 


%----------------------------------------------------------------------------------------
%	AWARDS - ACHIEVEMENTS SECTION
%----------------------------------------------------------------------------------------

\section{AWARDS/ \\ ACHIEVEMENTS} 

\begin{enumerate}
\item Recipient of Doctoral grant for PhD studies.
\item Recipient of University financial assistance award in Masters based on academic ranking.
\item Recipient of University scholarship in Bachelors, based on academic performance.
\item Recipient of Ministry of Science and Technology scholarship for undergrad studies, awarded to top 250 students throughout Pakistan.
\end{enumerate} 

%----------------------------------------------------------------------------------------
%	PUBLICATIONS SECTION
%----------------------------------------------------------------------------------------

\section{PUBLICATIONS} 
\begin{enumerate}
\item Integrating Sequence Evolution into Probabilistic Orthology Analysis -- accepted in {\sl Systematic Biology} (Impact factor 11.53)
\item Species tree inference using a mixture model -- {\sl Submitting to Molecular Biology and Evolution}
\item IThresholdPicker: An interactive threshold picker for performance evaluation in ROC based analysis -- {\sl in Manuscript}
\end{enumerate}


%----------------------------------------------------------------------------------------
%	PUBLICATIONS SECTION
%----------------------------------------------------------------------------------------

\section{REFERENCES} 
\begin{itemize}
\item Dr. Jens Lagergren \\
Professor in computer science and computational biology \\
KTH \& Scilifelab Stockholm \\
\href{http://www.nada.kth.se/~jensl/}{http://www.nada.kth.se/$\sim$jensl/} 

\item Dr. Lars Arvestad \\
Senior lecturer in computational biology \\
Stockholm University \& Scilifelab Stockholm \\
\hfill \href{http://www.nada.kth.se/~arve/}{http://www.nada.kth.se/$\sim$arve/}
\end{itemize}

%----------------------------------------------------------------------------------------

\end{resume}
\end{document}

