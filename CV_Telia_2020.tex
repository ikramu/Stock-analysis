%%%%%%%%%%%%%%%%%%%%%%%%%%%%%%%%%%%%%%%%%
% Medium Length Graduate Curriculum Vitae
% LaTeX Template
% Version 1.1 (9/12/12)
%
% This template has been downloaded from:
% http://www.LaTeXTemplates.com
%
% Original author:
% Rensselaer Polytechnic Institute (http://www.rpi.edu/dept/arc/training/latex/resumes/)
%
% Important note:
% This template requires the res.cls file to be in the same directory as the
% .tex file. The res.cls file provides the resume style used for structuring the
% document.
%
%%%%%%%%%%%%%%%%%%%%%%%%%%%%%%%%%%%%%%%%%

%----------------------------------------------------------------------------------------
%	PACKAGES AND OTHER DOCUMENT CONFIGURATIONS
%----------------------------------------------------------------------------------------

\documentclass[margin, 10pt]{res} % Use the res.cls style, the font size can be changed to 11pt or 12pt here

\usepackage{helvet} % Default font is the helvetica postscript font
%\usepackage{newcent} % To change the default font to the new century schoolbook postscript font uncomment this line and comment the one above

\usepackage{verbatim}
\usepackage{hyperref}
\hypersetup{
    colorlinks,
    linkcolor={green!50!black},
    citecolor={green!50!black},
    urlcolor={blue!80!black}
}
\usepackage[usenames,dvipsnames]{xcolor}
\usepackage[T1]{fontenc}

\setlength{\textwidth}{5in} % Text width of the document

\begin{document}

%----------------------------------------------------------------------------------------
%	NAME AND ADDRESS SECTION
%----------------------------------------------------------------------------------------

\moveleft.5\hoffset\centerline{\huge\textbf{\textsc{Ikram Ullah}} } % Your name at the top
 
\moveleft\hoffset\vbox{\hrule width\resumewidth height 1pt}\smallskip % Horizontal line after name; adjust line thickness by changing the '1pt'
 
\moveleft.5\hoffset\centerline{S\"odergatan 2B, Lgh 1405 \hfill \href{mailto:ikramu@kth.se}{ikramu@kth.se}} % Your address
\moveleft.5\hoffset\centerline{19534, M\"arsta \hfill \href{tel:+46739366175}{+46 72 598 8985}}%(073) 936-6175}
\moveleft.5\hoffset\centerline{Sweden \hfill \href{https://www.linkedin.com/in/ikramu/}{Linkedin Profile}}
%\moveleft.5\hoffset\centerline{}
%\moveleft.5\hoffset\centerline{}
\moveleft\hoffset\vbox{\hrule width\resumewidth height 1pt}\smallskip % Horizontal line after name; adjust line thickness by changing the '1pt'

%----------------------------------------------------------------------------------------

\begin{resume}

%----------------------------------------------------------------------------------------
%	OBJECTIVE SECTION
%----------------------------------------------------------------------------------------
 
\section{SUMMARY}
I have been working with machine learning and data science in planning, modeling, implementation, and analysis capacity for more than a decade. 
I have had the opportunity to work with data from different domains including Telecom, Life Sciences, Imaging, Stock Exchange and Call Center.
Apart from academic research in machine learning, I have experience of working with companies from different sectors like Telecom, IT and Energy.
I have worked as a independent freelancer, a team member and an area lead in teams of varying sizes.
%Since 2009, I have been working with big data at modeling, implementation, and analysis level in various capacities. The data, I have analyzed, came from different sources including stock exchange, call center, electricity consumption, and genomics (mainly DNA- and RNA-sequencing data). I am well-versed in some of the existing open source data analysis tools and have a desire to learn new concepts and technologies.
%\section{OBJECTIVE}  


%----------------------------------------------------------------------------------------
%	AWARDS - ACHIEVEMENTS SECTION
%----------------------------------------------------------------------------------------

\section{SELECTED \\ACHIEVE-\\-MENTS} 

\begin{enumerate}
 \item Started the focus area of data-driven cybersecurity at Ericsson. We have completed multiple projects and the team has expanded since its start
 \item Selected for Key Contributer award during first year at Ericsson on the basis of overall performance (Email attached at the end of the CV) 
 \item Regularly published in some of the top journals in my field during PhD and Postdoctoral career 
 \item Recipient of scholarships during entire academic career from bachelors to doctoral studies based on academic performance. 
 \item Recipient of Ministry of Science and Technology scholarship for undergrad studies, awarded to top 250 students throughout Pakistan.
\end{enumerate} 


%----------------------------------------------------------------------------------------
%	EDUCATION SECTION
%----------------------------------------------------------------------------------------

\section{EDUCATION}

{\bf \color{Black}{Doctor of Philosophy},} Computer Science \\
{\color{RubineRed}{Kungliga Tekniska H\"{o}gskolan (KTH), Sweden}} \hfill \textit{Jun 2010 -- March 2015} \\
Focus: Machine Learning \\
Designing Bayesian generative models and implementing them using Markov chain Monte Carlo (MCMC) and Expectation Maximization (EM) techniques. I applied these methods to infer evolutionary relationships among genes and species using next-generation DNA-sequencing data % from related genes and species.

%-----------------------------------------%
{\bf \color{Black}{Exchange Student},} Computer Science \\
{\color{RubineRed}{University of Limerick, Ireland}} \hfill \textit{Mar 2009 -- Aug 2009} \\
Focus: Software Engineering 

%-----------------------------------------%
{\bf \color{Black}{Master of Science,}} Computer Science \\ %(CGPA 3.47 out of 4.0) \\
{\color{RubineRed}{Lahore University of Management Sciences, Pakistan}} \hfill \textit{Sep 2006 -- Dec 2008} \\
Focus: Artificial Intelligence and Software Engineering \\
%Minor: Software Engineering

%-----------------------------------------%
{\bf \color{Black}{Bachelor of Science,}} Computer Science \\ % (CGPA 3.8 out of 4.0) \\
{\color{RubineRed}{University of Peshawar, Pakistan}} \hfill \textit{Jan 2002 -- Mar 2006} \\

%----------------------------------------------------------------------------------------
%	PROFESSIONAL EXPERIENCE SECTION 
%----------------------------------------------------------------------------------------

\section{PROFESSIONAL \\ EXPERIENCE}

{\sl \textbf{\color{Brown}{\textbf{Researcher SW Technology}}}} \hfill \textit{March 2018 onward} \\
{\color{RubineRed}{Ericsson AB, Sweden}} \\
%\href{http://www.ericsson.com}{Ericsson} is a Swedish multinational networking and telecommunications company headquartered in Stockholm. The company offers services, software and infrastructure in information and communications technology. 
I am technical lead for data-driven RAN security at Networks division in Ericsson. 
My responsibilities include strategy development, conducting product-driven research and developing proof-of-concepts 
in the following areas.

\begin{itemize}
    \item Detection and root cause analysis of air interface attacks (originating from both suspicious devices as well as Network)
    \item Generating/Identifying data and developing machine learning pipelines for cybersecurity use cases
    \item Developing strategy for data driven RAN security in collaboration with internal stakeholders
    \item Managing/advising team-members ranging from thesis student to senior data scientist
    \item Managing Swedish/EU collaborations related to data-driven cybersecurity projects with academia and industry
\end{itemize}

{\sl \textbf{\color{Brown}{\textbf{Postdoctoral Bioinformatician}}}} \hfill \textit{April 2015 -- March 2018} \\
{\color{RubineRed}{Radiumhemmet, Karolinska University Hospital}} \\
During my tenure, I drove my own research project (first one below) as well as assisted lab members in computational analysis of their projects. Some of the projects include:
\begin{itemize}
\item \textbf{Tumor evolution patterns in metastatic breast cancer}: Using evolutionary methods, we inferred the mode of tumor progression in metastatic breast cancer patients. We published the results in \href{https://www.jci.org/articles/view/96149}{Journal of Clinical Investigation} (Impact factor 12.28)
\item \textbf{Quantifying intra-metastatic heterogeneity in breast cancer}: Using DNA-sequencing data consisting of multiple blocks from the same metastasis, we quantified the extent of intra-metastatic heterogeneity and changing landscape of mutations in different regions of metastasis. We published the results in \href{https://bmccancer.biomedcentral.com/articles/10.1186/s12885-017-3815-2}{BMC Cancer} (Impact factor 3.42).
\item \textbf{Research collaborations and Teaching}: I actively collaborated with my team members and other research teams (both within and outside Sweden) in computational analysis of cancer (see one \href{https://bmccancer.biomedcentral.com/articles/10.1186/s12885-018-4021-6}{publication}). I also co-tought Bioinformatics course to undergrad students.
\end{itemize}


{\sl \textbf{\color{Brown}{\textbf{Chief Data Scientist (part-time)}}}} \hfill \textit{April 2014 -- Jan 2015} \\
{\color{RubineRed}{Greenely, Sweden}} \\
\href{http://www.greenely.com}{Greenely} was a KTH Innovation startup (now a established company) aiming to optimize household electricity consumption using user behavior mining and energy disaggregation algorithms. 
I was one of the early members of the team, responsible for:
\begin{itemize} 
\item Research about open source smart meter data and energy disaggregation algorithms
\item Master thesis supervision who was exploring various machine learning algorithms for energy disaggregation 
\item Visualization and reporting of electricity usage and related (e.g., weather) statistics for showing energy usage patterns
\end{itemize} 

{\sl \color{Brown}{\textbf{Analyst Software Engineer}}} \hfill \textit{Oct 2009 -- April 2010} \\
{\href{http://www.satmapinc.com/}{\color{RubineRed}{Afiniti (formerly TRG SATMAP), Machine Learning Team, Pakistan}}}  \\
SATMAP is an enterprise call center application used for intelligent call routing used by Fortune 500 companies like AT\&T, Gieco, Time Warners Cables etc. I was part of the machine learning team and my responsibilities included:

\begin{itemize} \itemsep -2pt % Reduce space between items
\item Evaluating different machine learning algorithms on call center data. 
\item Implementing, testing, and integrating selected algorithms in SATMAP. 
\end{itemize}
 
{\sl \textbf{\color{Brown}{\textbf{Machine learning freelancer}}}} \hfill \textit{2007 -- 2013} \\
{\href{https://www.upwork.com/freelancers/~013a228837c241737c?viewMode=1}{\color{RubineRed}{Upwork profile}}}  \\
I worked on industry-related problems in big data and machine learning. Some of the projects were in the areas of 
stock value forecasting, text cateogrization of legal documents and barcode image quality analysis
\end{itemize}

%----------------------------------------------------------------------------------------
%	COMPUTER SKILLS SECTION
%----------------------------------------------------------------------------------------

\section{COMPUTING \\ SKILLS} 

{\bf \color{Brown}{Languages:}} 
Python, R, C++ and Java\\
{\bf \color{Brown}{Data/Streaming pipelines:}} From time to time, I have used ELK stack, Kafka/Zookeeper, InfluxDB and Grafana for real-time visualizations of anomalous data \\
{\bf \color{Brown}{Misc Tools:}} Bash, Git, Netbeans, Eclipse, CMake, MySQL, SQL Server, RapidMiner, Boost C++ libraries, OpenMPI \& Boost-MPI \\
{\bf \color{Brown}{Cluster/Cloud infrastructure:}} I have worked with Swedish computing clusters at \href{https://www.pdc.kth.se/hpc-services/computing-systems}{KTH} for data-intensive simulations and analyses. 
Experience working with dockers and Amazon cloud services like EC2. Basic knowledge of cloud-native concepts \\
{\bf \color{Brown}{Operating Systems:}} (In order of preference) Unix/MacOS and Windows.

%----------------------------------------------------------------------------------------
%	Team leading SECTION
%---------------------------------------------------------------------------------------- 
\section{MACHINE LEARNING SKILLS}
{\bf \color{Brown}{ML areas:}} 
Bayesian generative models, Mixture models using Expectation maximization, Deep learning models, Probabilistic graphical models, Sampling methods based on MCMC, 
Discriminative methods like SVM, Clustering techniques, Association rules mining   \\
{\bf \color{Brown}{Statistical Theory:}} Statistical distributions and hypothesis testing, Statistical significance (confidence intervals, p and q-values), Parametric and non-parametric bootstrapping \\
{\bf \color{Brown}{Visualization and reporting:}} Different types of visualizations in Python (Matplotlib and Seaborn) and R (ggplot2), Kibana, Grafana
%----------------------------------------------------------------------------------------

%----------------------------------------------------------------------------------------
%	SOFTWARE SECTION
%----------------------------------------------------------------------------------------

\section{SW DEV \\SKILLS} 
Some of softwares I co-developed are briefly discussed below and available on my Bitbucket page \href{https://bitbucket.org/ikramu}{(https://bitbucket.org/ikramu)}.
\begin{itemize}
\item \textbf{Cancer evolution pipeline} -- A pipeline for inferring the life history of metastatic breast cancer using machine learning algorithms followed by its visualization using tree-visualization tools. Majority of the implementation is in R. Main algorithm is at my \href{https://bitbucket.org/ikramu/dolloparsimonyforcancerevolution}{Bitbucket page}. 
\item \textbf{DLRSOrthology} -- DLRSOrthology is a program for Bayesian probabilistic orthology analysis using an integrated gene evolution and sequence evolution model as explained in the \href{https://doi.org/10.1093/sysbio/syv044}{paper}. The implementation is in GNU C++ using Boost and PrIME libraries while the auxiliary scripts are written in R, Python and Perl. DLRSOrthology is available \href{https://bitbucket.org/ikramu/dlrsorthology}{here}.
\item \textbf{MixTreEM} -- It is implementation of MixTreEM: a species tree inference algorithm using mixture model as explained in the \href{https://doi.org/10.1093/molbev/msv115}{paper}. The software is writtten in C++ and Boost-MPI library is used for parallelizing the code. MixTreEM is available \href{https://bitbucket.org/ikramu/mixtreem}{here}.
\item \textbf{PrIME} -- PrIME is a C++ library with tools for phylogenetic inference. Computationally, the emphasis is on probabilistic models that employ a Markov Chain Monte Carlo (MCMC) technique to draw samples from the posterior distribution. The library is available \href{http://prime.scilifelab.se/}{here}.
\item \textbf{JPrIME} -- JPrIME is the Java-based implementation of PrIME with computational optimization of existing tools, and inclusion of additional tools for phylogenetic inference. The library is available \href{https://github.com/arvestad/jprime}{here}
\end{itemize}

%----------------------------------------------------------------------------------------
%	PUBLICATIONS SECTION
%----------------------------------------------------------------------------------------

\section{PUBLICATIONS} 
Check my \href{https://scholar.google.com/citations?user=HjM0XDoAAAAJ&hl=en}{Google scholar page}.


%----------------------------------------------------------------------------------------
%	RECOMMENDATION SECTION
%----------------------------------------------------------------------------------------

\section{RECOMMEND- \\-ATIONS } 
May be provided on request. Some may be found on my \href{https://www.linkedin.com/in/ikramu/}{Linkedin} page.  %Some personal and professional references have been furnished in the designated section in my application.

%----------------------------------------------------------------------------

\end{resume}
\end{document}