%%%%%%%%%%%%%%%%%%%%%%%%%%%%%%%%%%%%%%%%%
% Medium Length Graduate Curriculum Vitae
% LaTeX Template
% Version 1.1 (9/12/12)
%
% This template has been downloaded from:
% http://www.LaTeXTemplates.com
%
% Original author:
% Rensselaer Polytechnic Institute (http://www.rpi.edu/dept/arc/training/latex/resumes/)
%
% Important note:
% This template requires the res.cls file to be in the same directory as the
% .tex file. The res.cls file provides the resume style used for structuring the
% document.
%
%%%%%%%%%%%%%%%%%%%%%%%%%%%%%%%%%%%%%%%%%

%----------------------------------------------------------------------------------------
%	PACKAGES AND OTHER DOCUMENT CONFIGURATIONS
%----------------------------------------------------------------------------------------

\documentclass[margin, 10pt]{res} % Use the res.cls style, the font size can be changed to 11pt or 12pt here

\usepackage{helvet} % Default font is the helvetica postscript font
%\usepackage{newcent} % To change the default font to the new century schoolbook postscript font uncomment this line and comment the one above

\usepackage{verbatim}
\usepackage{hyperref}
\hypersetup{
    colorlinks,
    linkcolor={green!50!black},
    citecolor={green!50!black},
    urlcolor={blue!80!black}
}
\usepackage[usenames,dvipsnames]{xcolor}
\usepackage[T1]{fontenc}

\setlength{\textwidth}{5.1in} % Text width of the document

\begin{document}

%----------------------------------------------------------------------------------------
%	NAME AND ADDRESS SECTION
%----------------------------------------------------------------------------------------

\moveleft.5\hoffset\centerline{\huge\textbf{\textsc{Ikram Ullah}} } % Your name at the top

\moveleft\hoffset\vbox{\hrule width\resumewidth height 1pt}\smallskip % Horizontal line after name; adjust line thickness by changing the '1pt'
\moveleft.5\hoffset\centerline{{\bf Date of Birth:} 28th Feb 1984 \hfill {\bf Nationality:} Swedish} 
\moveleft.5\hoffset\centerline{{\bf Marital Status:} Married \hfill {\bf Email:} \href{mailto:ikramu@kth.se}{ikramu@kth.se}} 
\moveleft.5\hoffset\centerline{{\bf LinkedIn Profile:} \href{https://www.linkedin.com/in/ikramu/}{Click here} \hfill {\bf Phone No:} \href{tel:+46725988985}{+46 72 598 8985}} 

\moveleft\hoffset\vbox{\hrule width\resumewidth height 1pt}\smallskip % Horizontal line after name; adjust line thickness by changing the '1pt'
 
%\moveleft.5\hoffset\centerline{S\"odergatan 2B, Lgh 1405 \hfill \href{mailto:ikramu@kth.se}{ikramu@kth.se}} % Your address
%\moveleft.5\hoffset\centerline{19534, M\"arsta \hfill \href{tel:+46739366175}{+46 73 936 6175}}%(073) 936-6175}
%\moveleft.5\hoffset\centerline{Sweden \hfill \href{https://www.linkedin.com/in/ikramu/}{Linkedin Profile}}
%\moveleft.5\hoffset\centerline{}
%\moveleft.5\hoffset\centerline{}
%\moveleft\hoffset\vbox{\hrule width\resumewidth height 1pt}\smallskip % Horizontal line after name; adjust line thickness by changing the '1pt'

%----------------------------------------------------------------------------------------

\begin{resume}

%----------------------------------------------------------------------------------------
%	OBJECTIVE SECTION
%----------------------------------------------------------------------------------------
 
\section{SUMMARY}
My area of research is molecular evolution. My PhD research was related to inferring evolutionary relationships in genes and species using phylogenetic methods based on approximation algorithms. During my Postdoctoral studies, I used phylogenetic and related statistical methods to investigate tumor evolution patterns in metastatic breast cancer, intratumor heterogeneity and personalized cancer medicine. For this, I used high throughput DNA- and RNA-sequencing data, microarray data and drug screening assays. 
%Since 2009, I have been working with big data at modeling, implementation, and analysis level in various capacities. The data, I have analyzed, came from different sources including stock exchange, call center, electricity consumption, and genomics (mainly DNA- and RNA-sequencing data). I am well-versed in some of the existing open source data analysis tools and have a desire to learn new concepts and technologies.
%\section{OBJECTIVE}  

%Using my machine learning and big data analytics' skills to extract relevant micro- and macro-level insights from multiple large-scale data sources 
%Using my machine learning and big data analytics' skills to extract information from data, and then relevant insights from that information 
%To model and implement market-based eco-friendly solutions for challenging problems utilizing sound mathematical principles, latest software and hardware technologies.


%----------------------------------------------------------------------------------------
%	EDUCATION SECTION
%----------------------------------------------------------------------------------------

\section{EDUCATION}

{\bf \color{Black}{Doctor of Philosophy},} Bioinformatics \\
{\color{RubineRed}{Kungliga Tekniska H\"{o}gskolan (KTH), Sweden}} \hfill \textit{Jun 2010 -- March 2015} \\
Focus: Molecular Evolution  \\
Inferring evolutionary relationships among genes and species using next-generation DNA-sequencing data. To accomplish this, I designed solutions based on Bayesian generative models and implemented them using Markov chain Monte Carlo (MCMC) and Expectation Maximization (EM) methods.

%-----------------------------------------%
{\bf \color{Black}{Exchange Student},} Computer Science \\
{\color{RubineRed}{University of Limerick, Ireland}} \hfill \textit{Mar 2009 -- Aug 2009} 

%-----------------------------------------%
{\bf \color{Black}{Master of Science,}} Computer Science \\ %(CGPA 3.47 out of 4.0) \\
{\color{RubineRed}{Lahore University of Management Sciences, Pakistan}} \hfill \textit{Sep 2006 -- Dec 2008} \\
Focus: Artificial Intelligence and Software Engineering \\
%Minor: Software Engineering

%-----------------------------------------%
{\bf \color{Black}{Bachelor of Science,}} Computer Science \\ % (CGPA 3.8 out of 4.0) \\
{\color{RubineRed}{University of Peshawar, Pakistan}} \hfill \textit{Jan 2002 -- Mar 2006} \\

%----------------------------------------------------------------------------------------
%	PROFESSIONAL EXPERIENCE SECTION 
%----------------------------------------------------------------------------------------

\section{PROFESSIONAL \\ EXPERIENCE}

{\sl \textbf{\color{Brown}{\textbf{Postdoctoral Researcher}}}} \hfill \textit{April 2015 -- March 2018} \\
{\color{RubineRed}{Karolinska University Hospital, Stockholm, Sweden}} \\
Main research topics during my postdoctoral tenure includes:
\begin{itemize}
\item \textbf{Tumor evolution patterns in metastatic breast cancer}: Using phylogenetic methods, we inferred the mode of tumor progression in metastatic breast cancer patients. We demonstrated first ever genomic evidence of minimal role of axillary lymph node metastasis in seeding distant organ metastasis. We also discerned the landscape of mutational signatures active during lifetime of metastatic breast cancer. Our results are published in Journal of Clinical Investigation (Impact Factor 12.80) and can be accessed at \href{https://doi.org/10.1172/JCI96149}{https://doi.org/10.1172/JCI96149}.
\item \textbf{Quantifying intra-metastatic heterogeneity in breast cancer}: Using next generation DNA-sequencing data consisting of multiple blocks from the same metastasis, we quantified the extent of intra-metastatic heterogeneity and changing landscape of subclones in different regions of metastasis. We also showed the presence of clinically druggable mutations present specifically in metastasis. Further, we showed that genomic instability continues even during late stages of cancer. We are compiling results and intend to report them in high-impact clinical journal.
\item \textbf{Research collaborations}: I had active collaborations with my team members and other research teams (both within and outside Sweden) in computational studies related to breast cancer. The projects included histopathological image analysis, high throughput drug screening and in-situ measurement of drug target engagement.
%\item \textbf{Research collaborations}: I am collaborating with my team members and other research team (both within and outside Sweden) in basic and translational research in breast cancer. The projects are related to histopathological image analysis, high throughput drug screening and in-situ measurement of drug target engagement.
\end{itemize}

{\sl \textbf{\color{Brown}{\textbf{Researcher in Software Technology}}}} \hfill \textit{Mar 2018 -- present} \\
{\color{RubineRed}{Ericsson AB, Sweden}} \\
\href{http://www.greenely.com}{Ericsson} Ericsson is a Swedish multinational networking and telecommunications company headquartered in Stockholm. The company offers services, software and infrastructure in information and communications technology. I am part of the research team in Software group and my duties include:
\begin{itemize} 
\item Research in machine learning for anomaly and intrusion detection 
\item Contributing to Ericsson's machine learning strategy for long term leadership
\item Supervising master thesis students in cybersecurity and machine learning 
\end{itemize} 

{\sl \textbf{\color{Brown}{\textbf{Chief Data Scientist (part-time)}}}} \hfill \textit{April 2014 -- Jan 2015} \\
{\color{RubineRed}{Greenely, Sweden}} \\
\href{http://www.greenely.com}{Greenely} manages and optimizes household electricity consumption using user behavior mining and energy disaggregation algorithms. I was one of the early members of the team and was responsible for:
\begin{itemize} 
\item Research about open source smart meter data and energy disaggregation algorithms
\item Visualization and reporting of electricity usage and related (e.g., weather) statistics for showing energy usage patterns
\item Supervising master thesis students exploring various machine learning algorithms for energy disaggregation 
\item Main tools used were R, Python and PostgreSQL
\end{itemize} 

{\sl \color{Brown}{\textbf{Analyst Software Engineer}}} \hfill \textit{Oct 2009 -- April 2010} \\
{\href{http://www.satmapinc.com/}{\color{RubineRed}{Afiniti (formerly TRG SATMAP), Machine Learning Team, Pakistan}}}  \\
SATMAP is an enterprise call center application used for intelligent call routing used by Fortune 500 companies like AT\&T, Gieco, Time Warners Cables etc. I was part of the machine learning team and my responsibilities included:

\begin{itemize} \itemsep -2pt % Reduce space between items
\item Evaluating different machine learning algorithms on call center data. 
\item Implementing, testing, and integrating selected algorithms in SATMAP. 
\item Evaluation was performed using Rapidminer and implementation/tuning of selected algorithms was done in R.
\end{itemize}
 
{\sl \textbf{\color{Brown}{\textbf{Research Assistant}}}} \hfill \textit{Dec 2007 -- Oct 2008} \\
{\color{RubineRed}{Computer Science Department, LUMS, Pakistan}} 
\begin{itemize} 
\item Woking with \href{http://www.sciencedirect.com/science/article/pii/S0167642304000978}{XVCL}: an Java-based open source software providing component based Document-View Architecture support for developers. 
\end{itemize} 


%----------------------------------------------------------------------------------------
%	COMPUTER SKILLS SECTION
%----------------------------------------------------------------------------------------

\section{BIOINFORMA- \\ -TICS SKILLS} 

%{\sl \textbf{\color{Brown}{\textbf{Chief Data Scientist}}}}
{\bf \color{Brown}{Sequence alignment tools:}} 
 BLAST, MUSCLE, BWA \\
{\bf \color{Brown}{Phylogenetics tools:}} 
 PrIME, MrBayes, BEAST, PHYLIP \\
{\bf \color{Brown}{Variant calling tools:}} 
  MuTect, VarScan, Samtools \\
{\bf \color{Brown}{Cancer analysis tools/pipelines:}} 
  GATK and similar packages/pipelines in R/Bioconductor\\

\section{COMPUTING \\ SKILLS} 

%{\sl \textbf{\color{Brown}{\textbf{Chief Data Scientist}}}}
{\bf \color{Brown}{Languages:}} 
R, C++, Java, Python, Matlab\\
{\bf \color{Brown}{Misc Tools:}} Bash, CMake, RapidMiner, Weka, Boost C++ libraries, OpenMPI \& Boost-MPI, MySQL, SQL Server, Git, SVN, Netbeans, Eclipse \\
{\bf \color{Brown}{HPC Clusters:}} Using different Swedish supercomputing clusters, e.g., \href{https://www.pdc.kth.se/resources/computers/historical-computers/ferlin}{Ferlin}, \href{http://www.uppmax.uu.se/}{Tintin}, \href{https://www.nsc.liu.se/systems/triolith/}{Triolith}, and \href{http://www.hpc2n.umu.se/resources/abisko}{Abisko} for data-intensive simulations and analyses. Experience using Amazon cloud services like EC2 \\
{\bf \color{Brown}{Operating Systems:}} Unix, MacOS and Windows.

%----------------------------------------------------------------------------------------
%	Team leading SECTION
%---------------------------------------------------------------------------------------- 
\section{BIG DATA SKILLS}
{\bf \color{Brown}{Machine Learning Skills:}} 
Probabilistic graphical models, Bayesian generative models and their implementation using Markov chain Monte Carlo methods, Mixture models and related inference using Expectation Maximization, Discriminative methods like SVM, Classification, Clustering, Association rules mining,   \\
{\bf \color{Brown}{Statistical Theory:}} Statistical distributions and hypothesis theory, Statistical significance (confidence intervals, p and q-values), Parametric and non-parametric bootstrapping \\
{\bf \color{Brown}{Statistical Visualization and reporting:}} Different types of visualizations mainly using R ggplot2 library, Using matplotlib (when needed)
%----------------------------------------------------------------------------------------

%----------------------------------------------------------------------------------------
%	Computer Vision SECTION
%---------------------------------------------------------------------------------------- 
%\section{COMPUTER VISION \\ EXPERIENCE}
%{\sl \textbf{\color{Brown}{\textbf{Eye-based cursor movement}}}} \\
%Using computer vision algorithms to track iris movement in the eye and map it to drive cursor movement on the screen. %The software was implemented in C++ using Intel OpenCV libraries.  \\
%{\sl \textbf{\color{Brown}{\textbf{Classification of barcode images}}}} \\
%Using different image denoising algorithms implemented in Image Processing toolbox in Matlab to classify images to %translatable and non-translatable classes. The solution was used as pre-step in a barcode reader application.  

%----------------------------------------------------------------------------------------
%	Team leading SECTION
%---------------------------------------------------------------------------------------- 
%\section{TEAM-LEAD \\ EXPERIENCE}
%\begin{itemize}
%\item I have managed a small team of developers in Pakistan during my \href{https://www.odesk.com/users/~013a228837c241737c}{consultancy career}. Major responsibilities included task allocation, time allocation, and ensuring team collaboration.
%\item Recently, I am leading a group of KTH masters students in \href{https://www.youtube.com/watch?v=COsyhfX7Pes}{Greenely} project. Major responsibilities include task allocation, task synchronization, and team motivation.
%\end{itemize}
%----------------------------------------------------------------------------------------
%	Modeling SECTION
%---------------------------------------------------------------------------------------- 

%\section{COURSES}

%Some machine learning and big data-related courses that I took at graduate level at KTH are:
%\begin{itemize}
%\item High performance computing
%\item Advanced machine learning
%\item Graph theory and algorithms
%\item Elements of statistical learning
%\item Algorithmic bioinformatics
%\end{itemize} 

%----------------------------------------------------------------------------------------
%	RESEARCH SOFTWARE SECTION
%----------------------------------------------------------------------------------------

\section{SOFTWARES DEVELOPED} 
Some of softwares that I developed, or co-developed, are given below. Some of them are available at my Bitbucket page \href{https://bitbucket.org/ikramu}{(https://bitbucket.org/ikramu)}.
\begin{itemize}
\item \textbf{Cancer evolution pipeline} -- A pipeline for inferring the life history of metastatic breast cancer using machine learning algorithms followed by its visualization using tree-visualization tools. Majority of the implementation is in R. Main algorithm is at my \href{https://bitbucket.org/ikramu/dolloparsimonyforcancerevolution}{Bitbucket page}. 
\item \textbf{DLRSOrthology} -- DLRSOrthology is a program for Bayesian probabilistic orthology analysis using an integrated gene evolution and sequence evolution model as explained in the \href{https://doi.org/10.1093/sysbio/syv044}{paper}. The implementation is in GNU C++ using Boost and PrIME libraries while the auxiliary scripts are written in R, Python and Perl. DLRSOrthology is available \href{https://bitbucket.org/ikramu/dlrsorthology}{here}.
\item \textbf{MixTreEM} -- It is implementation of MixTreEM: a species tree inference algorithm using mixture model as explained in the \href{https://doi.org/10.1093/molbev/msv115}{paper}. The software is writtten in C++ and Boost-MPI library is used for parallelizing the code. MixTreEM is available \href{https://bitbucket.org/ikramu/mixtreem}{here}.
\item \textbf{PrIME} -- PrIME is a C++ library with tools for phylogenetic inference. Computationally, the emphasis is on probabilistic models that employ a Markov Chain Monte Carlo (MCMC) technique to draw samples from the posterior distribution. The library is available \href{http://prime.scilifelab.se/}{here}.
\item \textbf{JPrIME} -- JPrIME is the Java-based implementation of PrIME with computational optimization of existing tools, and inclusion of additional tools for phylogenetic inference. The library is available \href{https://github.com/arvestad/jprime}{here}
\end{itemize}

%----------------------------------------------------------------------------------------
%	TEACHING SKILLS 
%----------------------------------------------------------------------------------------

\section{TEACHING \& PRESENTATION SKILLS}
I have been regularly presenting my research to larger audience during my PhD and Postdoctoral career. Based on the feedback, I believe I have very good presentation skills. Some of my teaching experience is given below: \\

{\bf \color{Brown}{Molecular Oncology and Biostatistics, 2017}} \\
I co-taught the Bioinformatics part of the course with two other researchers. My duties included: 
\begin{itemize} 
\item Lecturing about introduction and analysis of genomics data 
\item Conducting lab exercises in analysis of cancer data
\end{itemize} 
{\bf \color{Brown}{Advanced Software Engineering, 2008}} \\
I was teaching assistant for the course, responsible for conducting lab exercises and marking assignments.

%----------------------------------------------------------------------------------------
%	PUBLICATIONS SECTION
%----------------------------------------------------------------------------------------

\section{SELECTED\\PUBLICATIONS} 
Full list of publications is available at \href{https://scholar.google.com/citations?user=HjM0XDoAAAAJ&hl=en}{Google scholar page}. Some selected publications are listed below.
\begingroup
\renewcommand{\section}[2]{}%
%\renewcommand{\chapter}[2]{}% for other classes
\begin{thebibliography}{}
\bibitem{ullah2018}
    Ikram Ullah, Govindasamy-Muralidharan Karthik, et al (2018). "Evolutionary history of metastatic breast cancer reveals minimal seeding from axillary lymph-nodes" {\sl The Journal of Clinical Investigation} (impact factor 12.784). Accessible at \href{https://doi.org/10.1172/JCI96149}{https://doi.org/10.1172/jci96149}
\bibitem{ullah2015}
    Ikram Ullah, Joel Sjostrand et al (2015). "Integrating sequence evolution into probabilistic orthology analysis." Systematic biology 64, no. 6 (2015): 969-982. (impact factor 8.523). Accessible at \href{https://doi.org/10.1093/sysbio/syv044}{https://doi.org/10.1093/sysbio/syv044}
\bibitem{ullah2015b}
    Ikram Ullah, Pekka Parviainen, Jens Lagergren (2015). Species tree inference using a mixture model. Molecular biology and evolution, 32(9), pp.2469-2482. (impact factor 10.217). Accessible at \href{https://doi.org/10.1093/molbev/msv115}{https://doi.org/10.1093/molbev/msv115}
\end{thebibliography}
\endgroup

%----------------------------------------------------------------------------------------
%	AWARDS - ACHIEVEMENTS SECTION
%----------------------------------------------------------------------------------------

\section{AWARDS/ \\ ACHIEVEMENTS} 

\begin{enumerate}
\item Recipient of Doctoral grant for PhD studies.
\item Recipient of University financial assistance award in Masters on the basis of academic ranking.
\item Recipient of University scholarship in Bachelors, based on academic performance.
\item Recipient of Ministry of Science and Technology scholarship for undergrad studies, awarded to top 250 students throughout Pakistan.
\end{enumerate} 

%----------------------------------------------------------------------------------------
%	PUBLICATIONS SECTION
%----------------------------------------------------------------------------------------

\section{RECOMMEND- \\-ATIONS AND \\ REFERENCES } 
Recommendations may be found on my \href{https://www.linkedin.com/in/ikramu/}{Linkedin} page. Some personal and professional references have been furnished in the designated section in my application.

%----------------------------------------------------------------------------

\end{resume}
\end{document}