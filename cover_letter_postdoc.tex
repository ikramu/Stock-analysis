\documentclass[margin, 10pt] {report}

\usepackage{url}
\begin{document}

\noindent
Dear Dr. Ander Andersson,

\noindent
I am a PhD student working with Prof. Jens Lagergren at Scilifelab and I will be defending my thesis in March 2015. 

\section*{Existing research}
My research area is in phylogenomics and, using NGS data, we have developed methods and tools for 
\begin{enumerate}
\item Probabilistic orthology analysis using species tree aware method. 
\item Species tree inference using a generative mixture model. 
\end{enumerate}

In the first project, we used a Bayesian model named DLRS for reconstructing and reconciling gene trees within a species tree. 
%The input consists of dated species tree and gene family data while the output consists of orthology estimates for the gene pairs. 
For all reconstructed gene trees, the speciation events for each gene pair were enumerated using a dynamic programming algorithm to infer the corresponding speciation probability. Due to the high dimensionality of underlying distribution, Markov Chain Monte Carlo (MCMC) was used to sample from the posterior distribution. More details may be found in the attached paper named orthology\_systematic\_biology.pdf (\textit{accepted in Systematic Biology}).

In the second project, we used a two phased approach for reconstructing a species tree in the presence of gene duplications and losses. 
%The input to first phase consists of gene family data and initial set k of (random) topologies. 
In the first phase, using an EM algorithm based on a mixture model, a set of $k$ maximum likelihood species tree topologies were reconstructed. 
%The input to the second phase are these reconstructed trees and the gene family data. 
In the second phase, using the DLRS model, the best species tree was selected by testing the viability of each gene family for each reconstructed species tree. More details may be found in the attached paper named mixtreem.pdf (\textit{submitted to Molecular Biology and Evolution}). 

\section*{Experience of high throughput data processsing}
%I would like to describe a bit about my experience with high throughput data. 
In the species tree project, I worked with mammalian genomic data downloaded from Ensembl. The tasks included pre-processing the single (around 3.5 GB) file containing the gene family data for 36 genomes and selecting the interesting families to reconstruct mammalian phylogeny. Since we have been working with generative models, I worked with large synthetic analysis, consisting of thousands of gene families, to verify our models. Lastly, as part of my industry experience, I worked with US demographic database for finding best caller/customer pair. For all these analyses, I used R, Java, bash, python and some machine learning tools like Rapidminer and Weka. For MCMC convergence analysis and day-today phylogenetic tasks in PhD, I use R mainly.    

I have experience working with parallel programming using MPI in C++. I used Boost MPI in my species tree reconstruction tool to significantly reduce the inference time for large data sets. 

\section*{Interest in current opportunity}
I enjoy mining big data for extracting interesting facts. In this regard, I am interested in devising approximate methods for high-dimensional problems using statistical machine learning techniques.
%based on machine learning principles. 
Regarding this opportunity, moving from genomic to meta-genomic research would be an exciting challenge for me both from a modeling and implementation point of view. I believe that I will have the opportunity to use my existing skills as well as learn new and related techniques for problem solving as well as implementation.

\section*{Contact and recommendations}
For more details, you may visit my page at KTH (\url{http://www.nada.kth.se/ikramu/}) or my LinkedIn page at (\url{http://se.linkedin.com/in/ikramu/}). I can furnish recommendation letters from my primary supervisor Prof Jens Lagergren and my co-supervisors Dr. Lars Arvestad and Dr. Bengt Sennblad. Since I am here at Scilifelab, we may meet anytime convenient to you. 

\hfill \break
Looking forward to your reply,
\hfill \break
Best regards \\
Ikram Ullah \\
+46 73 936 6175

\end{document}
