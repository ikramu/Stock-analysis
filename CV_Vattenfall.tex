%%%%%%%%%%%%%%%%%%%%%%%%%%%%%%%%%%%%%%%%%
% Medium Length Graduate Curriculum Vitae
% LaTeX Template
% Version 1.1 (9/12/12)
%
% This template has been downloaded from:
% http://www.LaTeXTemplates.com
%
% Original author:
% Rensselaer Polytechnic Institute (http://www.rpi.edu/dept/arc/training/latex/resumes/)
%
% Important note:
% This template requires the res.cls file to be in the same directory as the
% .tex file. The res.cls file provides the resume style used for structuring the
% document.
%
%%%%%%%%%%%%%%%%%%%%%%%%%%%%%%%%%%%%%%%%%

%----------------------------------------------------------------------------------------
%	PACKAGES AND OTHER DOCUMENT CONFIGURATIONS
%----------------------------------------------------------------------------------------

\documentclass[margin, 10pt]{res} % Use the res.cls style, the font size can be changed to 11pt or 12pt here

\usepackage{helvet} % Default font is the helvetica postscript font
%\usepackage{newcent} % To change the default font to the new century schoolbook postscript font uncomment this line and comment the one above

\usepackage{verbatim}
\usepackage{hyperref}
\hypersetup{
    colorlinks,
    linkcolor={green!50!black},
    citecolor={green!50!black},
    urlcolor={blue!80!black}
}
\usepackage[usenames,dvipsnames]{xcolor}
\usepackage[T1]{fontenc}

\setlength{\textwidth}{5.1in} % Text width of the document

\begin{document}

%----------------------------------------------------------------------------------------
%	NAME AND ADDRESS SECTION
%----------------------------------------------------------------------------------------

\moveleft.5\hoffset\centerline{\huge\textbf{\textsc{Ikram Ullah}} } % Your name at the top
 
\moveleft\hoffset\vbox{\hrule width\resumewidth height 1pt}\smallskip % Horizontal line after name; adjust line thickness by changing the '1pt'
 
\moveleft.5\hoffset\centerline{S\"odergatan 2B, Lgh 1405 \hfill \href{mailto:ikramu@kth.se}{ikramu@kth.se}} % Your address
\moveleft.5\hoffset\centerline{19534, M\"arsta \hfill \href{tel:+46739366175}{+46 73 936 6175}}%(073) 936-6175}
\moveleft.5\hoffset\centerline{Sweden \hfill \href{https://www.linkedin.com/in/ikramu/}{Linkedin Profile}}
%\moveleft.5\hoffset\centerline{}
%\moveleft.5\hoffset\centerline{}
\moveleft\hoffset\vbox{\hrule width\resumewidth height 1pt}\smallskip % Horizontal line after name; adjust line thickness by changing the '1pt'

%----------------------------------------------------------------------------------------

\begin{resume}

%----------------------------------------------------------------------------------------
%	OBJECTIVE SECTION
%----------------------------------------------------------------------------------------
 
\section{SUMMARY}
Since 2009, I have been working with big data at modeling, implementation, and analysis level in various capacities. The data, I have analyzed, came from different sources including stock exchange, call center, electricity consumption, and genomics (mainly DNA- and RNA-sequencing data). I am well-versed in some of the existing open source data analysis tools and have a desire to learn new concepts and technologies.
%\section{OBJECTIVE}  

%Using my machine learning and big data analytics' skills to extract relevant micro- and macro-level insights from multiple large-scale data sources 
%Using my machine learning and big data analytics' skills to extract information from data, and then relevant insights from that information 
%To model and implement market-based eco-friendly solutions for challenging problems utilizing sound mathematical principles, latest software and hardware technologies.


%----------------------------------------------------------------------------------------
%	EDUCATION SECTION
%----------------------------------------------------------------------------------------

\section{EDUCATION}

{\bf \color{Black}{Doctor of Philosophy},} Computer Science \\
{\color{RubineRed}{Kungliga Tekniska H\"{o}gskolan (KTH), Sweden}} \hfill \textit{Jun 2010 -- March 2015} \\
Focus: Machine Learning \\
Designing Bayesian generative models and implementing them using Markov chain Monte Carlo (MCMC) and Expectation Maximization (EM) techniques. I applied these methods to infer evolutionary relationships among genes and species using next-generation DNA-sequencing data % from related genes and species.

%-----------------------------------------%
{\bf \color{Black}{Exchange Student},} Computer Science \\
{\color{RubineRed}{University of Limerick, Ireland}} \hfill \textit{Mar 2009 -- Aug 2009} \\
Focus: Software Engineering 

%-----------------------------------------%
{\bf \color{Black}{Master of Science,}} Computer Science \\ %(CGPA 3.47 out of 4.0) \\
{\color{RubineRed}{Lahore University of Management Sciences, Pakistan}} \hfill \textit{Sep 2006 -- Dec 2008} \\
Focus: Artificial Intelligence and Software Engineering \\
%Minor: Software Engineering

%-----------------------------------------%
{\bf \color{Black}{Bachelor of Science,}} Computer Science \\ % (CGPA 3.8 out of 4.0) \\
{\color{RubineRed}{University of Peshawar, Pakistan}} \hfill \textit{Jan 2002 -- Mar 2006} \\

%----------------------------------------------------------------------------------------
%	PROFESSIONAL EXPERIENCE SECTION 
%----------------------------------------------------------------------------------------

\section{PROFESSIONAL \\ EXPERIENCE}

{\sl \textbf{\color{Brown}{\textbf{Postdoctoral Researcher}}}} \hfill \textit{April 2015 onward} \\
{\color{RubineRed}{Radiumhemmet, Karolinska Hospital}} \\
Using terabytes of DNA/RNA sequencing data from breast cancer patients to infer the route of cancer spread (metastasis) from primary tumor to other organs. 
\begin{itemize} 
\item Preprocessing of sequencing data to extract relevant cancer-specific variants like mutations and copy number aberrations
\item Using existing and implementing new statistical methods for forensic analysis of variants to infer evolutionary relationships
\item Visualization and reporting of results
\item Assisting team-members and collaborators in statistical analysis and visualization of their biomedical data 
\item Main tools used are R, Bash, Python and various bioinformatics softwares (in C++ and Java). 
\item Analysis is performed on super-computing clusters
\end{itemize} 

{\sl \textbf{\color{Brown}{\textbf{Chief Data Scientist (part-time)}}}} \hfill \textit{April 2014 -- Jan 2015} \\
{\color{RubineRed}{Greenely, Sweden}} \\
\href{http://www.greenely.com}{Greenely} manages and optimizes household electricity consumption using user behavior mining and energy disaggregation algorithms. I was one of the early members of the team and was responsible for:
\begin{itemize} 
\item Research about open source smart meter data and energy disaggregation algorithms
\item Visualization and reporting of electricity usage and related (e.g., weather) statistics for showing energy usage patterns
\item Supervising master thesis students exploring various machine learning algorithms for energy disaggregation 
\item Main tools used were R, Python and PostgreSQL
\end{itemize} 

{\sl \color{Brown}{\textbf{Analyst Software Engineer}}} \hfill \textit{Oct 2009 -- April 2010} \\
{\href{http://www.satmapinc.com/}{\color{RubineRed}{Afiniti (formerly TRG SATMAP), Machine Learning Team, Pakistan}}}  \\
SATMAP is an enterprise call center application used for intelligent call routing used by Fortune 500 companies like AT\&T, Gieco, Time Warners Cables etc. I was part of the machine learning team and my responsibilities included:

\begin{itemize} \itemsep -2pt % Reduce space between items
\item Evaluating different machine learning algorithms on call center data. 
\item Implementing, testing, and integrating selected algorithms in SATMAP. 
\item Evaluation was performed using Rapidminer and implementation/tuning of selected algorithms was done in R.
\end{itemize}

{\sl \textbf{\color{Brown}{\textbf{Online Freelancing}}}} \hfill \textit{2008 -- 2013, part-time} \\
{\color{RubineRed}{Upwork (formerly Odesk)} } \\
I started consulting at rentacoder.com (now acquired by freelancer.com) and then switched to odesk.com, offering consultancy mainly for machine learning projects. Some of the projects are the following: 
\begin{itemize} \itemsep -2pt % Reduce space between items
\item Stock value analysis and prediction using association rule mining (in R and Java)
\item Quality-based classification of barcode images using computer vision algorithms (in Matlab)
\item Legal text classification using different classification algorithms implemented in \href{https://rapidminer.com/}{Rapidminer} 
\end{itemize} 

 
{\sl \textbf{\color{Brown}{\textbf{Research Assistant}}}} \hfill \textit{Dec 2007 -- Oct 2008} \\
{\color{RubineRed}{Computer Science Department, LUMS, Pakistan}} 
\begin{itemize} 
\item Woking with \href{http://www.sciencedirect.com/science/article/pii/S0167642304000978}{XVCL}: an Java-based open source software providing component based Document-View Architecture support for developers. 
\end{itemize} 


%----------------------------------------------------------------------------------------
%	COMPUTER SKILLS SECTION
%----------------------------------------------------------------------------------------

\section{COMPUTING \\ SKILLS} 

%{\sl \textbf{\color{Brown}{\textbf{Chief Data Scientist}}}}
{\bf \color{Brown}{Languages:}} 
Mainly R, C++ and Java. Python and Matlab (when needed)\\
{\bf \color{Brown}{Misc Tools:}} RapidMiner, Weka, Bash, MySQL, PostgreSQL, CMake, Boost C++ libraries, OpenMPI \& Boost-MPI, Git, SVN, Netbeans, Eclipse, and different Bioinformatics tools and pipelines. \\
{\bf \color{Brown}{HPC Clusters:}} Using different Swedish Unix-based clusters, e.g., \href{https://www.pdc.kth.se/resources/computers/historical-computers/ferlin}{Ferlin}, \href{http://www.uppmax.uu.se/}{Tintin}, \href{https://www.nsc.liu.se/systems/triolith/}{Triolith}, and \href{http://www.hpc2n.umu.se/resources/abisko}{Abisko} for data-intensive simulations and analyses. Experience using Amazon cloud services like EC2 \\
{\bf \color{Brown}{Operating Systems:}} (In order of preference) Unix, MacOS and Windows.

%----------------------------------------------------------------------------------------
%	Team leading SECTION
%---------------------------------------------------------------------------------------- 
\section{BIG DATA SKILLS}
{\bf \color{Brown}{Machine Learning Skills:}} 
Probabilistic graphical models, Bayesian generative models and their implementation using Markov chain Monte Carlo methods, Mixture models and its inference using Expectation Maximization, Discriminative methods like SVM, Neural Networks, Classification, Clustering, Association rules mining,   \\
{\bf \color{Brown}{Statistical Theory:}} Statistical distributions and hypothesis theory, Statistical significance (confidence intervals, p and q-values), Parametric and non-parametric bootstrapping \\
{\bf \color{Brown}{Statistical Visualization and reporting:}} Different types of plots and visualizations mainly using R ggplot2 library and matplotlib (when needed)
%----------------------------------------------------------------------------------------

%----------------------------------------------------------------------------------------
%	Computer Vision SECTION
%---------------------------------------------------------------------------------------- 
%\section{COMPUTER VISION \\ EXPERIENCE}
%{\sl \textbf{\color{Brown}{\textbf{Eye-based cursor movement}}}} \\
%Using computer vision algorithms to track iris movement in the eye and map it to drive cursor movement on the screen. %The software was implemented in C++ using Intel OpenCV libraries.  \\
%{\sl \textbf{\color{Brown}{\textbf{Classification of barcode images}}}} \\
%Using different image denoising algorithms implemented in Image Processing toolbox in Matlab to classify images to %translatable and non-translatable classes. The solution was used as pre-step in a barcode reader application.  

%----------------------------------------------------------------------------------------
%	Team leading SECTION
%---------------------------------------------------------------------------------------- 
%\section{TEAM-LEAD \\ EXPERIENCE}
%\begin{itemize}
%\item I have managed a small team of developers in Pakistan during my \href{https://www.odesk.com/users/~013a228837c241737c}{consultancy career}. Major responsibilities included task allocation, time allocation, and ensuring team collaboration.
%\item Recently, I am leading a group of KTH masters students in \href{https://www.youtube.com/watch?v=COsyhfX7Pes}{Greenely} project. Major responsibilities include task allocation, task synchronization, and team motivation.
%\end{itemize}
%----------------------------------------------------------------------------------------
%	Modeling SECTION
%---------------------------------------------------------------------------------------- 

\section{COURSES}

Some machine learning and big data-related courses that I took at graduate level at KTH are:
\begin{itemize}
\item High performance computing
\item Advanced machine learning
\item Graph theory and algorithms
\item Elements of statistical learning
\item Algorithmic bioinformatics
\end{itemize} 

%----------------------------------------------------------------------------------------
%	RESEARCH SOFTWARE SECTION
%----------------------------------------------------------------------------------------

\section{SOFTWARES DEVELOPED} 
Some of softwares that I developed, or co-developed, are given below.
\begin{itemize}
\item \textbf{Cancer evolution pipeline} -- A pipeline for inferring the life history of metastatic breast cancer using machine learning algorithms followed by its visualization using tree-visualization tools. Majority of the implementation is in R. Main algorithm is at my \href{https://bitbucket.org/ikramu/dolloparsimonyforcancerevolution}{Bitbucket page}. 
\item \textbf{DLRSOrthology} -- DLRSOrthology is a program for Bayesian probabilistic orthology analysis using an integrated gene evolution and sequence evolution model as explained in the \href{https://doi.org/10.1093/sysbio/syv044}{paper}. The implementation is in GNU C++ using Boost and PrIME libraries while the auxiliary scripts are written in R, Python and Perl. DLRSOrthology is available \href{https://bitbucket.org/ikramu/dlrsorthology}{here}.
\item \textbf{MixTreEM} -- It is implementation of MixTreEM: a species tree inference algorithm using mixture model as explained in the \href{https://doi.org/10.1093/molbev/msv115}{paper}. The software is writtten in C++ and Boost-MPI library is used for parallelizing the code. MixTreEM is available \href{https://bitbucket.org/ikramu/mixtreem}{here}.
\item \textbf{PrIME} -- PrIME is a C++ library with tools for phylogenetic inference. Computationally, the emphasis is on probabilistic models that employ a Markov Chain Monte Carlo (MCMC) technique to draw samples from the posterior distribution. The library is available \href{http://prime.scilifelab.se/}{here}.
\item \textbf{JPrIME} -- JPrIME is the Java-based implementation of PrIME with computational optimization of existing tools, and inclusion of additional tools for phylogenetic inference. The library is available \href{https://github.com/arvestad/jprime}{here}
\end{itemize}

%----------------------------------------------------------------------------------------
%	AWARDS - ACHIEVEMENTS SECTION
%----------------------------------------------------------------------------------------

\section{AWARDS/ \\ ACHIEVEMENTS} 

\begin{enumerate}
\item Recipient of Doctoral grant for PhD studies.
\item Recipient of University financial assistance award in Masters on the basis of academic ranking.
\item Recipient of University scholarship in Bachelors, based on academic performance.
\item Recipient of Ministry of Science and Technology scholarship for undergrad studies, awarded to top 250 students throughout Pakistan.
\end{enumerate} 

%----------------------------------------------------------------------------------------
%	TEACHING SKILLS 
%----------------------------------------------------------------------------------------

\section{TEACHING \& PRESENTATION SKILLS}
I have been regularly presenting my work and novel research findings to larger audience during my PhD and Postdoctoral career. Based on the feedback, I believe I have very good presentation skills. I have also taught some courses at university level some of which are given below: \\

{\bf \color{Brown}{Molecular Oncology and Biostatistics, 2017}} \\
I co-taught the Bioinformatics part of the course with two other researchers. My duties included: 
\begin{itemize} 
\item Lecturing about introduction and analysis of genomics data 
\item Conducting lab exercises in analysis of cancer data
\end{itemize} 
{\bf \color{Brown}{Advanced Software Engineering, 2008}} \\
I was teaching assistant for the course, responsible for conducting lab exercises and marking assignments.


%----------------------------------------------------------------------------------------
%	PUBLICATIONS SECTION
%----------------------------------------------------------------------------------------

\section{PUBLICATIONS} 
My previous publications are listed on my \href{https://scholar.google.com/citations?user=HjM0XDoAAAAJ&hl=en}{Google scholar page}. The following papers are accepted and will soon be published.
\begingroup
\renewcommand{\section}[2]{}%
%\renewcommand{\chapter}[2]{}% for other classes
\begin{thebibliography}{}
\bibitem{ullah2017}
    \textbf{Ikram Ullah}, Govindasamy-Muralidharan Karthik, Amjad Alkodsi, Una Kj\"allquist, Gustav St\aa lhammar, John L\"ovrot, Nelson-Fuentes Martinez, Jens Lagergren, Sampsa Hautaniemi, Johan Hartman, and Jonas Bergh. \textit{Evolutionary history of metastatic breast cancer reveals minimal seeding from axillary lymph-nodes} -- accepted in {\sl Journal of Clinical Investigation (JCI)} (impact factor 12.784)
\bibitem{una2017}
    Una Kj\"allquist, Rikard Erlandsson, Nicholas P Tobin, Amjad Alkodsi, \textbf{Ikram Ullah}, Gustav St\aa lhammar, Eva Karlsson, Thomas Hatschek, Johan Hartman, Sten Linnarsson, Jonas Bergh. \textit{Exome sequencing of primary breast cancers with paired metastatic lesions reveals metastasis-enriched mutations in the A-kinase anchoring protein family (AKAPs)} -- accepted in {\sl BMC Cancer} (impact factor 3.288)
\bibitem{karthik2017}
    Karthik Govindasamy Muralidharan, Mattias Rantalainen, Gustav St\aa lhammar, John L\"ovrot, \textbf{Ikram Ullah}, Amjad Alkodsi, Ran Ma, Lena Wedlund, Johan Lindberg, Jan Frisell, Jonas Bergh, Johan Hartman. \textit{Intra-tumor heterogeneity in breast cancer has limited impact on transcriptomic-based molecular profiling} -- accepted in {\sl BMC Cancer} (impact factor 3.288)
\end{thebibliography}
\endgroup


%----------------------------------------------------------------------------------------
%	PUBLICATIONS SECTION
%----------------------------------------------------------------------------------------

\section{RECOMMEND- \\-ATIONS AND \\ REFERENCES } 
Recommendations may be found on my \href{https://www.linkedin.com/in/ikramu/}{Linkedin} and Upwork page. References may be furnished on request
\begin{comment}
\begin{itemize}
\item Dr. Jens Lagergren \\
Professor in computer science and computational biology \\
KTH \& Scilifelab Stockholm \\
\href{http://www.nada.kth.se/~jensl/}{http://www.nada.kth.se/$\sim$jensl/} 

\item Dr. Lars Arvestad \\
Senior lecturer in computational biology \\
Stockholm University \& Scilifelab Stockholm \\
\hfill \href{http://www.nada.kth.se/~arve/}{http://www.nada.kth.se/$\sim$arve/}
\end{itemize}
\end{comment}
%----------------------------------------------------------------------------------------

\end{resume}
\end{document}