%%%%%%%%%%%%%%%%%%%%%%%%%%%%%%%%%%%%%%%%%
% Medium Length Graduate Curriculum Vitae
% LaTeX Template
% Version 1.1 (9/12/12)
%
% This template has been downloaded from:
% http://www.LaTeXTemplates.com
%
% Original author:
% Rensselaer Polytechnic Institute (http://www.rpi.edu/dept/arc/training/latex/resumes/)
%
% Important note:
% This template requires the res.cls file to be in the same directory as the
% .tex file. The res.cls file provides the resume style used for structuring the
% document.
%
%%%%%%%%%%%%%%%%%%%%%%%%%%%%%%%%%%%%%%%%%

%----------------------------------------------------------------------------------------
%	PACKAGES AND OTHER DOCUMENT CONFIGURATIONS
%----------------------------------------------------------------------------------------

\documentclass[margin, 10pt]{res} % Use the res.cls style, the font size can be changed to 11pt or 12pt here

\usepackage{helvet} % Default font is the helvetica postscript font
%\usepackage{newcent} % To change the default font to the new century schoolbook postscript font uncomment this line and comment the one above

\usepackage{hyperref}
\usepackage[usenames,dvipsnames]{xcolor}

\setlength{\textwidth}{5.1in} % Text width of the document

\begin{document}

%----------------------------------------------------------------------------------------
%	NAME AND ADDRESS SECTION
%----------------------------------------------------------------------------------------

\moveleft.5\hoffset\centerline{\large\bf Ikram Ullah} % Your name at the top
 
\moveleft\hoffset\vbox{\hrule width\resumewidth height 1pt}\smallskip % Horizontal line after name; adjust line thickness by changing the '1pt'
 
\moveleft.5\hoffset\centerline{Studentbacken 21/1515} % Your address
\moveleft.5\hoffset\centerline{11557, Stockholm}
\moveleft.5\hoffset\centerline{(073) 936-6175}

%----------------------------------------------------------------------------------------

\begin{resume}

%----------------------------------------------------------------------------------------
%	OBJECTIVE SECTION
%----------------------------------------------------------------------------------------
 
\section{OBJECTIVE}  

To model and implement market-based eco-friendly solutions for real world problems utilizing sound mathematical principles, with the help of latest hardware and software technologies.

%To model and implement mathematically inspired eco-friendly solutions for real world phenomena, with a focus on market-based solutions, using latest hardware and software.

%----------------------------------------------------------------------------------------
%	PROFESSIONAL EXPERIENCE SECTION
%----------------------------------------------------------------------------------------
 
\section{EXPERIENCE}

{\sl Statistician/Mathematician} \hfill Oct 2009 -- May 2010 \\
SATMAP Inc, Machine Learning Department, Karachi, Pakistan  \\
SATMAP is an enterprise call center application used for intelligent call routing. I worked with a team of data scientist to:

\begin{itemize} \itemsep -2pt % Reduce space between items
\item Evaluating newer machine learning/data mining algorithms on call center data. 
\item Implementing, testing, and integrating them with SATMAP. 
\end{itemize}
 
{\sl Consulting/Freelancing} \hfill On-and-off \\
Odesk \hfill \href{https://www.odesk.com/users/~013a228837c241737c}{https://www.odesk.com/users/$\sim$013a228837c241737c} \\
I consulted mainly for machine learning projects. Some of these include: 

\begin{itemize} \itemsep -2pt % Reduce space between items
\item Stock value analysis and prediction using association rule mining
\item Text classification using different machine learning algorithms in Rapidminer 
\item Barcode scanning using computer vision algorithms
\end{itemize} 
 
{\sl Research Assistant} \hfill Dec 2007 – Oct 2008 \\
Computer Science Department, Lahore University of Management Sciences. 
\begin{itemize} 
\item Woking with XVCL: an open source component library that provides component based Document/View Architecture support for developers.
\item Teaching assistant for software engineering course
\end{itemize} 

{\sl Software Internee} \hfill Summers 2007 \\
Five Rivers Technologies, Lahore
\begin{itemize}
\item Part of the group implementing a dual streaming/synchronization server for mobile devices.
\end{itemize} 

%----------------------------------------------------------------------------------------
%	Modeling SECTION
%---------------------------------------------------------------------------------------- 

\section{MODELING \\ SKILLS}

Over the course of my academic/industry career, I have had the opportunity to work in multiple modeling/algorithmic areas. Some of them include:  \\
\begin{itemize}
\item Probabilistic graphical models especially Bayesian hierarchical models
\item Combanitorial tree search algorithms based on dynamic programming
\item Unsupervised learning especially mixture models
\item Support Vector Machines 
\item Markov Chain Monte Carlo, especially Metropolis-based sampling methods
\item Expectation Maximization
\end{itemize} 

%----------------------------------------------------------------------------------------
%	COMPUTER SKILLS SECTION
%----------------------------------------------------------------------------------------

\section{COMPUTER \\ SKILLS} 

{\bf Languages:} 
C++, Java, R, Matlab, C\#,  \\
{\bf Misc Tools:} Bash, CMake, Python, (Some) Boost C++ libraries, OpenMPI, MySQL, SQL Server, RapidMiner, Weka, Git, SVN, Netbeans, Eclipse, and different Bioinformatics tools. \\
{\bf Operating Systems:} Unix, Windows and OS X.  \\
{\bf HPC Clusters:} Worked on 4 Swedish Unix-based super computers (Ferlin, Triolith, Tintin, and Abisko)

%----------------------------------------------------------------------------------------
%	EDUCATION SECTION
%----------------------------------------------------------------------------------------

\section{EDUCATION}

{\bf Doctor of Philosophy,} Computer Science \\
Kungliga Tekniska Högskolan, Sweden. expected December Oct 2014 \\
Concentration: Machine Learning \\
Minor: Genetics 

%-----------------------------------------%
{\bf Master of Science,} Computer Science \\
Lahore University of Management Sciences, Pakistan. Sep 2006 -- Aug 2009 \\
Concentration: Machine Learning \\
Minor: Software Engineering

%-----------------------------------------%
{\bf Exchange Student,} Computer Science \\
University of Limerick, Ireland. Mar 2009 -- Aug 2009 \\
Concentration: Software Engineering 

%-----------------------------------------%
{\bf Bachelor of Science,} Computer Science \\
University of Peshawar, Pakistan. Jan 2002 -- Jan 2006 \\
Concentration: Computer Science 

%----------------------------------------------------------------------------------------
%	AWARDS - ACHIEVEMENTS SECTION
%----------------------------------------------------------------------------------------

\section{AWARDS \\ ACHIEVEMENTS} 

\begin{enumerate}
\item Recipient of Doctoral grant for PhD studies.
\item Recipient of University financial assistance award in Masters.
\item Recipient of University scholarship in Bachelors, based on academic performance.
\item Recipient of Ministry of Science and Technology scholarship for undergrad studies, awarded to top 250 students throughout Pakistan.
\end{enumerate} 

%----------------------------------------------------------------------------------------
%	PUBLICATIONS SECTION
%----------------------------------------------------------------------------------------

\section{PUBLICATIONS} 

Integrating Sequence Evolution into Probabilistic Orthology Analysis (under review in {\sl Systematic Biology} impact factor 11.53\\
Species tree inference using a mixture model (in {\sl Manuscript})

%----------------------------------------------------------------------------------------
%	PUBLICATIONS SECTION
%----------------------------------------------------------------------------------------

\section{REFERENCES} 
Prof. Jens Lagergren \hfill \href{http://www.nada.kth.se/~jensl/}{http://www.nada.kth.se/$\sim$jensl/} \\
Dr. Lars Arvestad \hfill \href{http://www.nada.kth.se/~arve/}{http://www.nada.kth.se/$\sim$arve/}

%----------------------------------------------------------------------------------------

\end{resume}
\end{document}

